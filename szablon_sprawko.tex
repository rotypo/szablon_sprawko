\documentclass[polish,a4paper,11pt]{mwart}

\usepackage[polish, english]{babel}
\usepackage[utf8]{inputenc}
\usepackage{polski}
\usepackage[T1]{fontenc}
\usepackage{indentfirst}
\frenchspacing

\usepackage{enumerate}
\usepackage{graphicx}
\usepackage{float}
\usepackage{makecell}
\usepackage{siunitx}
\sisetup{output-decimal-marker = {,}}
\usepackage{icomma}
\let\lll\undefined
\usepackage{amsmath, amssymb, amsfonts}
\usepackage{mathtools}
\usepackage{import}		% wklejanie pdf_tex
\usepackage{xcolor}		% kolory

\usepackage[%			% bibliografia
style=numeric,
sorting=none,
language=autobib,
autolang=other,
backend=biber,
]{biblatex}

\usepackage{csquotes}
\DeclareQuoteAlias{croatian}{polish}

\addbibresource{bibliografia.bib}

\title{Tytuł}
\author{Autor}
\date{Data}

\begin{document}

	\maketitle

	lub

	\begin{table}[h] %Tabelka
	\centering
		\begin{tabular}{ | c |  >{\centering\arraybackslash}m{5.5cm} | c | }
			\hline
			\makecell{ \textbf{Wydział:} \\ IMiR \\ \textbf{Rok:}~3 \\ Semestr: 6 } &
			\textbf{\large{Przedmiot}} &
			\makecell{Data \\ wykonania \\ ćwiczenia: \\ data} \\ \hline
			\makecell{\emph{Wykonujący ćw.:} \\ osoba \\ osoba } &
			\large{Temat} &
			\makecell{Nr ćwiczenia: \\ 1} \\ \hline
		\end{tabular}
	\end{table}

	\printbibliography
	
\end{document}